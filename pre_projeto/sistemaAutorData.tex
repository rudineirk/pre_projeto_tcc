\subsubsection{Sistema Autor Data}

A indicação da fonte pode ser feita de duas formas:

\begin{itemize}
  \item Entre parênteses, pelo sobrenome do autor, ou pela instituição responsável ou pelo título de entrada em letras maiúsculas seguido da data de publicação do documento, separados por vírgula.

  Exemplo: Num estudo recente \cite{Trainotti2014} é exposto...

  \item   Incluso na sentença, pelo sobrenome do autor, ou pela instituição responsável ou pelo título de entrada em letras maiúsculas e minúsculas, indicando-se a data entre parênteses.

  Exemplo: Segundo \citeonline{Trainotti2014}, "a presença de concreções de bauxita no Rio Cricon...".

  \item Transcrições com mais de três linhas - devem ser destacadas com recuo de 4 cm da margem esquerda, em letra menor que a utilizada no texto, sem aspas, com a referência consultada no final do texto.

  Exemplo : A teleconferência permite ao indivíduo participar de um encontro nacional ou regional sem a necessidade de deixar seu local de origem. Tipos comuns de teleconferência incluem o uso de televisão, telefone, computador. Através de áudio, conferência, utilizando a companhia local de telefone, um sinal de áudio pode ser emitido em um salão de qualquer dimensão \cite{Trainotti2014}.

  \item Citação de informação extraída da internet. Deve-se indicar os dados que possibilitem sua identificação, e incluí-la na lista de referências.

  Exemplos : Costumo dizer que o problema do ano 2000 é ao mesmo tempo de fácil e de difícil solução \cite{Internet2014}.

\end{itemize}
