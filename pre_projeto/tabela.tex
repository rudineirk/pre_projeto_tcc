\subsection{Tabela}

Estes elementos devem ser auto-explicativos, apresentam informações tratadas, conforme normas de apresentação tabular do Instituto Brasileiro de Geografia e Estatística - IBGE. Possui forma não discursiva de apresentação de informações e tem por objetivo a descrição e/ou o cruzamento de dados numéricos, codificações,especificações técnicas e símbolos.

\begin{table}[htbp]
\centering
\caption{Exemplo de tabela}
\label{tab:exTabela}
\begin{tabular}{|l|l|} \hline
  Modelo & Característica \\
  \hline \hline
  \citeonline{NonakaTakeuchi1997}  & Criação          \\
  \citeonline{DevenportPrusak1998} & Compartilhamento \\
  \citeonline{Sveiby1998}          & Compartilhamento \\
  \citeonline{Bartol2002}          & Criação       \\
  \citeonline{Gagne2009}           & Compartilhamento \\
  \citeonline{Imani2007}           & Criação         \\
  \hline
\end{tabular}
\\ \footnotesize Fonte: o autor
\end{table}

Devem conter: numeração independente e progressiva em algarismos arábicos; título na parte superior, precedido da palavra Tabela (fonte menor que a do texto); as referências citadas deverão aparecer como nota d e rodapé; não cabendo em uma folha, a tabela deverá continuar em outra repetindo o título, cabeçalho e outras informações utilizadas na primeira. Preferencialmente, deverão ser alinhadas às margens laterais do texto e, quando pequenas, centralizadas. Podem ser intercaladas no texto ou em apêndice.

\begin{table}[htbp]
\centering
\caption{Testando a segunda tabela}
\label{tab:exTabela}
\begin{tabular}{|l|l|} \hline
  Modelo & Característica \\
  \hline \hline
  \citeonline{NonakaTakeuchi1997}  & Criação          \\
  \citeonline{DevenportPrusak1998} & Compartilhamento \\
  \hline
\end{tabular}
\\ \footnotesize Fonte: \citeonline{AutorDesconhecido2014}
\end{table}
