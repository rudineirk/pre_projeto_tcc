\subsection{Figuras}

Estes elementos devem ser auto-explicativos, explicam ou complementam visualmente o texto: quadros, lâminas, plantas, fotografias, gráficos, organogramas, fluxogramas, esquemas, desenhos e outros.

Exemplo: Quadros - elementos que contêm informações qualitativas, normalmente textuais e dados não estatísticos.

\begin{figure}[htbp]
\centering
\includegraphics[width=0.2\textwidth]{imagens/exemplo.jpg}
\caption{Exemplo de figura}
\label{fig:exemploFig1}
\footnotesize Fonte: Baseado em \citeonline{Trainotti2014}
\end{figure}

Sua identificação aparece na parte inferior, precedida da palavra designativa, seguida de seu numero de ordem de ocorrência no texto, em algarismos arábicos, do respectivo título/ e ou legenda explicativa de forma breve e clara, dispensando consulta ao texto, e da fonte. A ilustração deve ser inserida o mais próximo possível do trecho a que se refere.

\begin{figure}[htbp]
\centering
\includegraphics[width=0.2\textwidth]{imagens/unisociesc.jpg}
\caption{Unisociesc}
\label{fig:exemploFig1}
\footnotesize Fonte: Baseado em O autor
\end{figure}
