\section{Justificativa}

  Softwares de gerenciamento de infraestrutura não são famosos por terem
  arquiteturas bem definidas e adequadas, segundo \cite{Tobin2006}, isto
  se deve a práticas das equipes de infraestrutura de criar \textit{scripts}
  que realizam a solução imediata do problema, mas dificultam muito a
  manutenção por não aplicarem práticas adequadas de desenvolvimento.

  Em uma interface com uma grande quantidade de regras de negócio aplicadas,
  a manutenção da mesma pode ter um custo reduzido se aplicados corretamente
  conceitos de arquitetura de software e isolamento de domínios de aplicação.
  Uma das formas de estruturar aplicações desta forma é a criação de
  microserviços.
