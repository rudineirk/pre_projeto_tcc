\section{Definição das Hipóteses de Trabalho}

  Segundo \citeonline{Fowler2016}, arquiteturas de microserviços reduzem
  alguns dos problemas provenientes do aumento da complexidade de uma aplicação
  por meio da divisão da aplicação em pequenos serviços com uma camada leve
  de comunicação entre si.

  Ao contrário de aplicações tradicionais utilizando arquitetura
  \textit{Service Oriented Architecture} (SOA), que expõe
  \textit{Remote Procedure Calls} (RPCs) detalhados e simulam o funcionamento
  de uma arquitetura monolítica \cite{Erl2008}, são expostas
  \textit{Application Programming Interfaces} (APIs) no estilo
  \textit{REpresentational State Transfer} (REST) que facilitam o uso por
  outras aplicações e por terceiros.

  O isolamento de domínios da aplicação diminui a inércia do fluxo de entrega
  de software funcional, uma vez que podem ser feitas entregas pequenas e
  rápidas, diminuindo assim o \textit{Time to Market}.
