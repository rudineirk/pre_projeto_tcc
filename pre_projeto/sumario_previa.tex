\section{Sumário}

  \noindent
  \textbf{1 INTRODUÇÃO} \\
  \textbf{2 ARQUITETURAS DE SOFTWARE} \\
  2.1 MONOLÍTICA \\
  2.2 SOA \\
  \textbf{2.2.1 SOAP} \\
  2.3 MICROSERVIÇOS \\
  \textbf{3 MICROSERVIÇOS} \\
  3.1 REST \\
  3.2 FORMATOS DE DADOS \\
  \textbf{3.2.1 XML} \\
  \textbf{3.2.2 JSON} \\
  3.3 MESSAGE QUEUES \\
  \textbf{3.3.1 Brokered} \\
  \textbf{3.3.2 Brokerless} \\
  3.3 BALANCEAMENTO DE CARGA \\
  3.4 DESCOBERTA DE SERVIÇOS \\
  \textbf{3.4.1 DNS} \\
  \textbf{3.4.2 Zookeeper} \\
  3.5 DOCKER \\
  3.6 ORQUESTRAÇAO DE MICROSERVIÇOS \\
  \textbf{3.6.1 Kubernetes} \\
  \textbf{4 BANCO DE DADOS} \\
  4.1 NoSQL \\
  4.2 DOCUMENTOS JSON \\
  \textbf{4.2.1 MongoDB} \\
  4.3 BIGTABLE \\
  \textbf{4.3.1 Cassandra} \\
  4.4 KEY-VALUE \\
  \textbf{4.4.1 Redis} \\
  \textbf{5 INTERFACE DE ADMINISTRAÇÃO DE SERVIDORES MODULAR} \\
  5.1 ARQUITETURA \\
  5.2 ROTINAS DE GERENCIAMENTO \\
  5.3 GERAÇÃO DE RELATÓRIOS \\
  \textbf{6 ROTINAS DE GERENCIAMENTO} \\
  6.1 CADASTROS \\
  \textbf{6.1.1 Gerenciamento integrado de filiais} \\
  6.2 GRÁFICOS EM TEMPO REAL \\
  6.3 STATUS DO SISTEMA \\
  \textbf{6.3.1 Envio de alertas} \\
  \textbf{7 GERAÇÃO DE RELATÓRIOS} \\
  7.1 PRÉ-PROCESSAMENTO DOS DADOS \\
  \textbf{7.1.1 MapReduce} \\
  7.2 GERAÇÃO INSTANTÂNEA \\
  7.3 GERAÇÃO AGENDADA \\
  \textbf{8 ARMAZENAMENTO DE DADOS} \\
  8.1 DADOS CADASTRAIS \\
  \textbf{8.1.1 Auditoria de operações} \\
  8.2 DADOS OPERACIONAIS \\
  \textbf{8.2.1 Processamento de logs} \\
  CONCLUSÃO \\
  REFERÊNCIAS \\
